\documentclass[a4paper,twoside]{article}
\usepackage[T1]{fontenc}
\usepackage[bahasa]{babel}
\usepackage{graphicx}
\usepackage{graphics}
\usepackage{float}
\usepackage[cm]{fullpage}
\pagestyle{myheadings}
\usepackage{etoolbox}
\usepackage{setspace} 
\usepackage{lipsum} 
\setlength{\headsep}{30pt}
\usepackage[inner=2cm,outer=2.5cm,top=2.5cm,bottom=2cm]{geometry} %margin
% \pagestyle{empty}

\makeatletter
\renewcommand{\@maketitle} {\begin{center} {\LARGE \textbf{ \textsc{\@title}} \par} \bigskip {\large \textbf{\textsc{\@author}} }\end{center} }
\renewcommand{\thispagestyle}[1]{}
\markright{\textbf{\textsc{AIF401/AIF402 \textemdash Rencana Kerja Skripsi \textemdash Sem. Genap 2020/2021}}}

\newcommand{\HRule}{\rule{\linewidth}{0.4mm}}
\renewcommand{\baselinestretch}{1}
\setlength{\parindent}{0 pt}
\setlength{\parskip}{6 pt}

\onehalfspacing
 
\begin{document}

\title{\@judultopik}
\author{\nama \textendash \@npm} 

%tulis nama dan NPM anda di sini:
\newcommand{\nama}{Nicholas Aditya Halim}
\newcommand{\@npm}{2017730018}
\newcommand{\@judultopik}{Implementasi Editor Kode pada SharIF Judge} % Judul/topik anda
\newcommand{\jumpemb}{1} % Jumlah pembimbing, 1 atau 2
\newcommand{\tanggal}{31/03/2021}

% Dokumen hasil template ini harus dicetak bolak-balik !!!!

\maketitle

\pagenumbering{arabic}

\section{Deskripsi}
SharIF Judge adalah modifikasi dari aplikasi Sharif Judge buatan Mohammad Javad Naderi yang digunakan pada beberapa kuliah di Informatika Unpar. Mahasiswa dapat mengunggah kode program, dan SharIF Judge akan menilai kebenaran program secara otomatis. 

Adanya pandemi COVID-19 menyebabkan seluruh kegiatan kuliah harus dilaksanakan secara daring. Kuliah daring menyebabkan adanya kesulitan untuk mengawasi kecurangan yang mungkin dilakukan mahasiswa saat ujian berlangsung. Untuk memudahkan pengawasan, diperlukan sebuah cara untuk merekam tindakan mahasiswa selama ujian berlangsung. Hal ini dapat dicapai dengan mengimplementasikan IDE pada SharIF Judge.

{\it Integrated Development Environment} (IDE) adalah sebuah aplikasi yang menyediakan fasilitas untuk pembangunan perangkat lunak. Sebuah IDE pada umumnya dapat digunakan untuk mengedit, mengompilasi, dan menjalankan kode.

Pada skripsi ini, akan diimplementasikan IDE pada SharIF Judge.  Dengan menggunakan IDE berbasis web ini, tindakan mahasiswa selama ujian seperti ketikan dapat direkam, sehingga memudahkan pengawasan selama ujian. 


\section{Rumusan Masalah}
Rumusan masalah yang akan dibahas pada skripsi ini adalah sebagai berikut:
\begin{itemize}
	\item Bagaimana mengimplementasikan {\it Integrated Development Environment} sehingga mahasiswa dapat mengetik dan menjalankan kode dalam SharIF Judge?
\end{itemize}


\section{Tujuan}
Tujuan dari topik skripsi ini adalah sebagai berikut:
\begin{itemize}
	\item Mengimplementasikan {\it Integrated Development Environment} sehingga mahasiswa dapat mengetik dan menjalankan kode dalam SharIF Judge.
\end{itemize}

\section{Deskripsi Perangkat Lunak}
Perangkat lunak akhir yang akan dibuat memiliki fitur minimal sebagai berikut:
\begin{itemize}
	\item Menjalankan seluruh fitur SharIF Judge yang sudah ada sebelumnya
	\item Mengetik kode program pada halaman web
	\item Menjalankan kode program dengan input tes kasus sendiri
\end{itemize}

\section{Detail Pengerjaan Skripsi}
Bagian-bagian pekerjaan skripsi ini adalah sebagai berikut :
\begin{enumerate}
	\item Melakukan studi mengenai komponen yang diperlukan untuk membuat IDE berbasis web
	\item Mempelajari struktur SharIF Judge
	\item Merancang IDE berbasis web untuk SharIF Judge
	\item Mengimplementasikan IDE pada SharIF Judge
	\item Melakukan pengujian dan eksperimen
	\item Menulis dokumen skripsi
\end{enumerate}

\section{Rencana Kerja}
Rincian capaian yang direncanakan di Skripsi 1 adalah sebagai berikut:
\begin{enumerate}
	\item Melakukan studi mengenai komponen yang diperlukan untuk membuat IDE berbasis web
	\item Mempelajari struktur SharIF Judge
	\item Mengimplementasikan IDE pada SharIF Judge
	\item Menulis bab 1-3 dokumen skripsi
\end{enumerate}

Sedangkan yang akan diselesaikan di Skripsi 2 adalah sebagai berikut:
\begin{enumerate}
	\item Melakukan pengujan dan eksperimen
	\item Menyempurnakan bab 1-3 dokumen skripsi
	\item Menulis bab 4-6 dokumen skripsi
\end{enumerate}

\vspace{1cm}
\centering Bandung, \tanggal\\
\vspace{2cm} \nama \\ 
\vspace{1cm}

Menyetujui, \\
\ifdefstring{\jumpemb}{2}{
\vspace{1.5cm}
\begin{centering} Menyetujui,\\ \end{centering} \vspace{0.75cm}
\begin{minipage}[b]{0.45\linewidth}
% \centering Bandung, \makebox[0.5cm]{\hrulefill}/\makebox[0.5cm]{\hrulefill}/2013 \\
\vspace{2cm} Nama: \makebox[3cm]{\hrulefill}\\ Pembimbing Utama
\end{minipage} \hspace{0.5cm}
\begin{minipage}[b]{0.45\linewidth}
% \centering Bandung, \makebox[0.5cm]{\hrulefill}/\makebox[0.5cm]{\hrulefill}/2013\\
\vspace{2cm} Nama: \makebox[3cm]{\hrulefill}\\ Pembimbing Pendamping
\end{minipage}
\vspace{0.5cm}
}{
% \centering Bandung, \makebox[0.5cm]{\hrulefill}/\makebox[0.5cm]{\hrulefill}/2013\\
\vspace{2cm} Nama: \makebox[3cm]{\hrulefill}\\ Pembimbing Tunggal
}
\end{document}


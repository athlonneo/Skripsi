%versi 3 (18-12-2016)
\chapter{Kode Program View}
\label{lamp:C}

%terdapat 2 cara untuk memasukkan kode program
% 1. menggunakan perintah \lstinputlisting (kode program ditempatkan di folder yang sama dengan file ini)
% 2. menggunakan environment lstlisting (kode program dituliskan di dalam file ini)
% Perhatikan contoh yang diberikan!!
%
% untuk keduanya, ada parameter yang harus diisi:
% - language: bahasa dari kode program (pilihan: Java, C, C++, PHP, Matlab, C#, HTML, R, Python, SQL, dll)
% - caption: nama file dari kode program yang akan ditampilkan di dokumen akhir
%
% Perhatian: Abaikan warning tentang textasteriskcentered!!
%

\lstinputlisting[language=diff, label ={lampA:Submit.twig}, caption= Perubahan pada \texttt{Submit.twig}]{./Lampiran/Diff/Submit.twig.diff}
\lstinputlisting[language=JavaScript, label ={lampA:shj_submit.js}, caption= \textit{File} baru  \texttt{shj\_submit.js}]{./Lampiran/Diff/shj_submit.js}
\lstinputlisting[language=CSS, label ={lampA:submit.css}, caption=  \textit{File} baru \texttt{submit.css}]{./Lampiran/Diff/submit.css}


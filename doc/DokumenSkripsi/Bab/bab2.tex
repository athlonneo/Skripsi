\chapter{Landasan Teori}
\label{chap:teori}

\section{CodeIgniter}
\label{sec:codeigniter} 

CodeIgniter adalah sebuah \textit{framework} untuk membangun situs web menggunakan PHP. Tujuan utamanya adalah untuk mempercepat pembuatan proyek dengan menyediakan \textit{library} yang lengkap untuk fungsi-fungsi yang umum digunakan, serta antarmuka yang sederhana dan struktur yang logis untuk mengakses \textit{library} tersebut.


\begin{figure}[H]
	\centering  
	\includegraphics[scale=0.3]{ci-flowchart}  
	\caption[\textit{Flow Chart} CodeIgniter]{\textit{Flow Chart} CodeIgniter} 
	\label{fig:ciflowchart} 
\end{figure} 

Gambar~\ref{fig:ciflowchart} mengilustrasikan bagaimana data mengalir pada sistem CodeIgniter.

\begin{enumerate}
	\item \textit{File} index.php berfungsi sebagai \textit{front controller}, menginisialisasi \textit{resource} utama untuk menjalankan CodeIgniter.
	\item Router meneliti \textit{request} HTTP dan menentukan apa yang harus dilakukan.
	\item Jika terdapat \textit{file} \textit{cache}, maka langsung dikirimkan ke \textit{browser}.
	\item Sebelum \textit{controller} dimuat, seluruh \textit{request} HTTP dan data dari user disaring terlebih dahulu untuk keamanan.
	\item \textit{Controller} memuat \textit{model}, \textit{library} utama, dan  \textit{resource} lainnya yang diperlukan.
	\item \textit{View} akhir lalu dikirim ke browser untuk dilihat. \textit{Cache} akan dibuat terlebih dahulu bila diaktifkan. 
\end{enumerate}

\subsection{Model-View-Controller}
\label{subs:cimvc} 
CodeIgniter menggunakan pola arsitektur MVC (\textit{Model-View-Controller}) sebagai dasarnya. MVC memisahkan proses logika aplikasi dari presentasi. Dengan demikian, halaman web dapat memuat sedikit \textit{script} karena presentasinya terpisah dari \textit{scripting} PHP.

\subsubsection{Model}
\textit{Model} merepresentasikan struktur data. Biasanya \textit{model} memiliki fungsi-fungsi yang membantu dalam mengambil, memasukkan, dan memperbarui informasi pada \textit{database}. Pada CodeIgniter, \textit{model} adalah sebuah kelas yang mengekstensi \verb|CI_Model| dan terletak di direktori \verb|application/models/|.

\begin{lstlisting}[language=php, caption=Contoh \textit{model}, label=kode:cimodel]
class Blog_model extends CI_Model {

        public $title;
        public $content;
        public $date;

        public function get_last_ten_entries()
        {
                $query = $this->db->get('entries', 10);
                return $query->result();
        }

        public function insert_entry()
        {
                $this->title    = $_POST['title']; // please read the below note
                $this->content  = $_POST['content'];
                $this->date     = time();

                $this->db->insert('entries', $this);
        }

        public function update_entry()
        {
                $this->title    = $_POST['title'];
                $this->content  = $_POST['content'];
                $this->date     = time();

                $this->db->update('entries', $this, array('id' => $_POST['id']));
        }

}
\end{lstlisting}

Kode \ref{kode:cimodel} merupakan contoh sebuah kelas \textit{model} pada CodeIgniter. Kelas tersebut mengekstensi \verb|CI_Model| dan memiliki fungsi untuk mengambil, memasukkan, dan memperbarui \textit{database}.
	
\subsubsection{View}
\textit{View} adalah informasi yang ditampilkan kepada pengguna. Pada CodeIgniter, \textit{view} merupakan sebuah halaman web atau sebagian dari halaman web yang terletak di direktori \verb|application/view/|.

\begin{lstlisting}[language=php, caption=Contoh \textit{view}, label=kode:ciview]
<html>
<head>
        <title>My Blog</title>
</head>
<body>
        <h1>Welcome to my Blog!</h1>
</body>
</html>
\end{lstlisting}

Kode \ref{kode:ciview} merupakan contoh sebuah \textit{view}. \textit{View} pada CodeIgniter harus dipanggil melalui \textit{Controller} dan tidak pernah dipanggil secara langsung.
	
\subsubsection{Controller}
\textit{Controller} adalah perantara dari \textit{model} dan \textit{view}, serta \textit{resource} lainnya yang diperlukan untuk memproses \textit{request} HTTP dan menghasilkan sebuah halaman web. Pada CodeIgniter, \textit{controller} adalah sebuah kelas yang mengekstensi \verb|CI_Controller| dan terletak di direktori \verb|application/controllers/|.

\begin{lstlisting}[language=php, caption=Contoh \textit{controller}, label=kode:cicontroller]
<?php
class Blog extends CI_Controller {

        public function index()
        {
                echo 'Hello World!';
        }

        public function comments()
        {
                echo 'Look at this!';
        }
}
\end{lstlisting}

Kode \ref{kode:cimodel} merupakan contoh sebuah kelas \textit{controller} pada CodeIgniter. Kelas tersebut mengekstensi \verb|CI_Controller| dan memiliki fungsi \verb|index()| dan \verb|comments()|. Fungsi \verb|index()| akan dipanggil secara otomatis jika tidak ada fungsi lain yang dipanggil. 

\begin{lstlisting}[language=php, caption=Contoh memuat \textit{model} dan menampilkan \textit{view}, label=kode:cimodelview]
<?php
class Blog_controller extends CI_Controller {

        public function blog()
        {
                $this->load->model('blog');

                $data['query'] = $this->blog->get_last_ten_entries();

                $this->load->view('blog', $data);
        }
}
}
\end{lstlisting}

Pada CodeIgniter, \textit{model} dan \textit{view} hanya dapat dimuat melalui controller. Pada contoh kode \ref{kode:cimodelview}, fungsi \verb|blog()| pada \textit{controller} memuat \textit{model} untuk mengambil data dari \textit{database}, lalu menampilkan \textit{view} yang memuat data tersebut. 

\subsection{URL CodeIgniter}
\label{subs:ciurl}

URL pada CodeIgniter menggunakan \textit{segment-based approach} yang dirancang untuk lebih mudah dibaca oleh \textit{search engine} dan manusia. Berikut ini adalah contoh sebuah URL pada CodeIgniter:
\begin{center}
    \verb|example.com/class/function/ID|    
\end{center}
\begin{itemize}
	\item Bagian pertama, \verb|class| merepresentasikan kelas \textit{controller} yang akan dipanggil.
	\item Bagian kedua,  \verb|function| merepresentasikan fungsi yang akan dipanggil.
	\item Bagian ketiga dan seterusnya, \verb|ID| merepresentasikan variabel yang akan digunakan.
\end{itemize}

\section{Twig}
\label{sec:twig}

Twig adalah sebuah \textit{template engine} untuk PHP. Sebuah \textit{template} Twig memuat \textit{variable} atau \textit{expression} yang nantinya akan diubah menjadi \textit{value} saat template dievaluasi, serta \textit{tag} yang mengontrol logika template.

\begin{lstlisting}[language=php, caption=Contoh template Twig, label=kode:twig]
<!DOCTYPE html>
<html>
    <head>
        <title>My Webpage</title>
    </head>
    <body>
        <ul id="navigation">
        
            <li><a href="{{ item.href }}">{{ item.caption }}</a></li>
        
        </ul>

        <h1>My Webpage</h1>
        {{ a_variable }}
    </body>
</html>
\end{lstlisting}

Kode \ref{kode:twig} merupakan contoh sebuah template Twig. Terdapat dua jenis \textit{delimiter}, yaitu \verb|| dan \verb|{{ ... }}|. \textit{Delimiter} \verb|| digunakan untuk menjalankan \textit{statement} seperti \verb|for| dan \verb|if|, sementara \textit{delimiter} \verb|{{ ... }}| digunakan untuk menampilkan nilai dari \textit{variable} atau \textit{expression}.

\section{PDF.js}
\label{sec:pdfjs} 
PDF.js adalah sebuah library JavaScript yang berfungsi untuk menampilkan \textit{file} Portable Document Format (PDF) menggunakan HTML5 \textit{Canvas}. PDF.js terdiri dari 3 \textit{layer}:

\begin{itemize}
	\item \textit{\textbf{Core}} merupakan bagian dimana proses \textit{parse} dan \textit{interpret} dilakukan terhadap \textit{binary} PDF.
	\item \textit{\textbf{Display}} mengambil \textit{layer} \textit{core} sebagai API yang lebih mudah digunakan untuk menampilkan PDF dan mengambil informasi lainnya dari sebuah dokumen.
	\item \textit{\textbf{Viewer}} membangun \textit{layer} \textit{display} sebagai halaman website dengan \textit{user interface} yang dapat ditampilkan di browser.
\end{itemize}

Salah satu cara untuk menampilkan \textit{file} PDF menggunakan PDF.js adalah dengan \textit{embed} layer \textit{viewer} yang sudah tersedia melalui \verb|web/viewer.js| pada sebuah \verb|iframe|. Contoh kode untuk \textit{embed} PDF.js untuk menampilkan file \verb|sample.pdf| dapat dilihat pada kode \ref{kode:pdfjs}.

\begin{lstlisting}[language=php, caption=Contoh kode untuk menggunakan PDF.js, label=kode:pdfjs]
<!DOCTYPE html>
<html>
    <iframe src="/web/viewer.html?file=sample.pdf"></iframe>
</html>
\end{lstlisting}

\section{Ace}
\label{sec:ace} 
Ace adalah sebuah library JavaScript yang berfungsi sebagai \textit{code editor}.
Ace memiliki fitur-fitur yang dapat ditemukan di \textit{code editor} pada umumnya. Kode \ref{kode:ace} merupakan contoh kode untuk menempatkan editor Ace pada \verb|div| dengan id \verb|editor|. Terdapat berbagai konfigurasi pada Ace editor, pada contoh ini digunakan tema \textit{monokai} dan mode \textit{syntax highlighting} untuk JavaScript.

\begin{lstlisting}[language=php, caption=Contoh kode untuk menggunakan Ace, label=kode:ace]
<!DOCTYPE html>
<html>
<head>
<title>ACE in Action</title>
</head>
<body>

<div id="editor">
function foo(items) {
    var x = "All this is syntax highlighted";
    return x;
}
</div>
    
<script src="/ace-builds/src-noconflict/ace.js" type="text/javascript" charset="utf-8"></script>
<script>
    var editor = ace.edit("editor");
    editor.setTheme("ace/theme/monokai");
    editor.session.setMode("ace/mode/javascript");
</script>
</body>
</html>
\end{lstlisting} 

Berikut ini beberapa fungsi Ace yang digunakan:
\begin{itemize}
    \item \verb|setTheme()| \\ Mengubah tema editor kode.
    \item \verb|setMode()| \\ Mengubah mode \textit{syntax highlighting} untuk bahasa pemrograman.
    \item \verb|setValue()| \\ Mengubah isi editor kode.
    \item \verb|getValue()| \\ Mengambil isi editor kode.
    \item \verb|setReadOnly()| \\ Mengatur editor kode menjadi \textit{read only}.
\end{itemize}




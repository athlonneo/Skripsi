%versi 2 (8-10-2016) 
\chapter{Pendahuluan}
\label{chap:intro}
   
\section{Latar Belakang}
\label{sec:label}

SharIF Judge (dengan IF kapital) adalah modifikasi dari aplikasi Sharif Judge buatan Mohammad Javad Naderi yang berfungsi untuk menilai kode program yang diunggah secara otomatis berdasarkan kunci jawaban yang disediakan. SharIF Judge digunakan pada beberapa kuliah di Informatika Unpar untuk mempermudah proses pengumpulan dan penilaian kode program, terutama saat ujian.

Dengan adanya situasi pandemi, seluruh kegiatan kuliah wajib dilaksanakan secara daring. Hal ini menyebabkan berbagai kesulitan, terutama dalam pelaksanaan ujian. Saat ujian sedang berlangsung, umumnya terdapat pengawas yang mengawasi mahasiswa secara fisik untuk mencegah kecurangan. Namun, pengawasan saat ujian menjadi sangat sulit untuk dilakukan saat kuliah dilaksanakan secara daring. Diperlukan sebuah cara untuk merekam tindakan-tindakan mahasiswa selama ujian daring berlangsung. 

Maka, pada skripsi ini akan diimplementasikan editor kode pada SharIF Judge, yang sudah memiliki kemampuan untuk mengompilasi dan menjalankan kode. Dengan demikian, SharIF Judge dapat menjadi sebuah {\it Integrated Development Environment} yang mampu memfasilitasi seluruh proses pembuatan kode serta merekamnya.

{\it Integrated Development Environment} (IDE) adalah sebuah aplikasi yang menyediakan fasilitas untuk pembangunan perangkat lunak. Sebuah IDE memiliki kemampuan untuk mengedit, mengompilasi, dan menjalankan kode program. Pada umumnya, mahasiswa menggunakan aplikasi IDE seperti Netbeans untuk membuat kode program yang kemudian diunggah ke SharIF Judge untuk dinilai.

Dengan mengimplementasikan IDE berbasis web pada SharIF Judge, pengawasan terhadap mahasiswa saat ujian dapat dipermudah dengan merekam ketikan pada editor kode dan mendeteksi bila mahasiswa sedang melihat aplikasi selain SharIF judge.


\section{Rumusan Masalah}
\label{sec:rumusan}
Rumusan masalah yang akan dibahas pada skripsi ini adalah sebagai berikut:
\begin{itemize}
	\item Bagaimana mengimplementasikan {\it Integrated Development Environment} sehingga mahasiswa dapat mengetik dan menjalankan kode dalam SharIF Judge?
\end{itemize}


\section{Tujuan}
\label{sec:tujuan}
Tujuan yang ingin dicapai skripsi ini adalah sebagai berikut:
\begin{itemize}
	\item Mengimplementasikan {\it Integrated Development Environment} sehingga mahasiswa dapat mengetik dan menjalankan kode dalam SharIF Judge.
\end{itemize}

\section{Batasan Masalah}
\label{sec:batasan}
Batasan masalah pada skripsi ini adalah sebagai berikut:
\begin{enumerate}
	\item Perangkat lunak diuji pada mata kuliah Dasar-dasar Pemrograman.
\end{enumerate}

\section{Metodologi}
\label{sec:metlit}
Metodologi pengerjaan skripsi ini adalah sebagai berikut:
\begin{enumerate}
	\item Melakukan studi mengenai komponen yang diperlukan untuk membuat IDE berbasis web.
	\item Mempelajari struktur SharIF Judge.
	\item Merancang IDE berbasis web untuk SharIF Judge.
	\item Mengimplementasikan IDE pada SharIF Judge.
	\item Melakukan pengujian dan eksperimen.
	\item Menulis dokumen skripsi.
\end{enumerate}

\section{Sistematika Pembahasan}
\label{sec:sispem}
Sistematika pembahasan skripsi ini adalah sebagai berikut:
\begin{itemize}
	\item Bab 1 Pendahuluan membahas mengenai latar belakang, rumusan masalah, tujuan, batasan masalah, metodologi, dan sistematika pembahasan.
	\item Bab 2 Landasan Teori
\end{itemize}

\chapter{Analisis}
\label{chap:analisis}

\section{Analisis Sistem Kini}
\label{sec:3:analisiskini} 

SharIF Judge adalah sebuah \textit{online judge} dengan fungsi utama untuk mengevaluasi kode program yang dikumpulkan oleh pengguna secara otomatis. SharIF Judge digunakan pada beberapa mata kuliah pemrograman Teknik Informatika Unpar untuk mempermudah proses pengumpulan dan penilaian kode program. Antarmuka web Sharif Judge dibangun menggunakan PHP dengan \textit{framework} CodeIgniter, disertai \textit{backend} menggunakan Bash.

\subsection{Fitur SharIF Judge}
\label{subs:3:fitur}

Halaman dan tampilan yang tersedia pada pengguna SharIF Judge bergantung pada \textit{role} akun yang digunakan pengguna tersebut. Pada bagian ini, \textit{role} akun yang digunakan adalah \textit{admin}. Berikut ini adalah halaman yang terdapat pada SharIF Judge dengan fitur dan kegunaannya:

\subsubsection{Dashboard}
    \begin{figure}[H]
    	\centering  
    	\includegraphics[scale=0.4]{Page/dashboard.PNG}  
    	\caption{Halaman Dashboard}
    	\label{fig:3:dashboard} 
    \end{figure} 
    
    Gambar \ref{fig:3:dashboard} menunjukkan halaman Dashboard. Pada halaman ini terdapat kalender yang menunjukkan durasi setiap \textit{assignment} dan daftar notifikasi.
    
\subsubsection{Settings}
    \begin{figure}[H]
    	\centering  
    	\includegraphics[scale=0.4]{Page/settings.PNG}  
    	\caption{Halaman Settings}
    	\label{fig:3:settings} 
    \end{figure} 
    
    Gambar \ref{fig:3:settings} menunjukkan halaman Settings. Pada halaman ini terdapat berbagai pengaturan yang ada pada SharIF Judge.
    
\subsubsection{Users}
    \begin{figure}[H]
    	\centering  
    	\includegraphics[scale=0.4]{Page/users.PNG}  
    	\caption{Halaman Users}
    	\label{fig:3:users} 
    \end{figure} 
    
    Gambar \ref{fig:3:users} menunjukkan halaman Users. Pada halaman ini terdapat \textit{list} seluruh pengguna yang terdaftar pada SharIF Judge. Pengguna juga dapat membuat, mengubah, dan menghapus pengguna.
    
\subsubsection{Notifications}
    \begin{figure}[H]
    	\centering  
    	\includegraphics[scale=0.4]{Page/notifications.PNG}  
    	\caption{Halaman Notifications}
    	\label{fig:3:notifications} 
    \end{figure} 
    
    Gambar \ref{fig:3:notifications} menunjukkan halaman Notifications. Pada halaman ini terdapat \textit{list} seluruh notifikasi. Pengguna juga dapat membuat, mengubah, dan menghapus notifikas.

\subsubsection{Assignments}
    \begin{figure}[H]
    	\centering  
    	\includegraphics[scale=0.4]{Page/assignments.PNG}  
    	\caption{Halaman Assignments}
    	\label{fig:3:assignments} 
    \end{figure} 
    
    Gambar \ref{fig:3:assignments} menunjukkan halaman Assignments. Pada halaman ini terdapat \textit{list} seluruh \textit{assignment}. Pengguna juga dapat membuat, mengubah, dan menghapus \textit{assignment}. Salah satu \textit{assignment} pada halaman ini harus dipilih untuk dapat menggunakan beberapa fitur lainnya pada SharIF Judge. Soal dalam bentuk PDF juga dapat diunduh melalui halaman ini. 
    
\subsubsection{Problems}
    \begin{figure}[H]
    	\centering  
    	\includegraphics[scale=0.4]{Page/problems.PNG}  
    	\caption{Halaman Problems}
    	\label{fig:3:problems} 
    \end{figure} 
    
    Gambar \ref{fig:3:problems} menunjukkan halaman Problems. Pada halaman ini terdapat detil dari setiap \textit{problem} dari \textit{assignment} yang dipilih. Pengguna juga dapat mengunggah file untuk dikumpulkan sebagai \textit{submission} untuk \textit{problem} yang dipilih.
    
\subsubsection{Submit}
    \begin{figure}[H]
    	\centering  
    	\includegraphics[scale=0.4]{Page/submit.PNG}  
    	\caption{Halaman Submit}
    	\label{fig:3:submit} 
    \end{figure} 
    
    Gambar \ref{fig:3:submit} menunjukkan halaman Submit. Pada halaman ini, pengguna dapat mengunggah file untuk dikumpulkan sebagai \textit{submission} dari \textit{problem} yang dipilih.

\subsubsection{Final Submissions}
    \begin{figure}[H]
    	\centering  
    	\includegraphics[scale=0.4]{Page/final_submissions.PNG}  
    	\caption{Halaman Final Submissions}
    	\label{fig:3:final_submissions} 
    \end{figure} 
    
    Gambar \ref{fig:3:final_submissions} menunjukkan halaman Final Submissions. Pada halaman ini, terdapat \textit{list} seluruh \textit{final submission} untuk \textit{assignment} yang dipilih. Pengguna juga dapat melihat \textit{file} atau kode yang diunggah, dan nilai yang didapatkannya.
    
\subsubsection{All Submissions}
    \begin{figure}[H]
    	\centering  
    	\includegraphics[scale=0.4]{Page/all_submissions.PNG}  
    	\caption{Halaman All Submissions}
    	\label{fig:3:all_submissions} 
    \end{figure} 
    
    Gambar \ref{fig:3:all_submissions} menunjukkan halaman All Submissions. Pada halaman ini, terdapat \textit{list} seluruh \textit{submission} untuk \textit{assignment} yang dipilih. Pengguna juga dapat melihat \textit{file} atau kode yang diunggah, dan nilai yang didapatkannya. Untuk setiap \textit{problem}, sebuah \textit{submission} dapat dipilih sebagai \textit{final submission} melalui halaman ini.
    
\subsubsection{Scoreboard}
    \begin{figure}[H]
    	\centering  
    	\includegraphics[scale=0.4]{Page/scoreboard.PNG}  
    	\caption{Halaman Scoreboard}
    	\label{fig:3:scoreboard} 
    \end{figure} 
    
    Gambar \ref{fig:3:scoreboard} menunjukkan halaman Scoreboard. Pada halaman ini, terdapat \textit{list} nilai pengguna untuk setiap \textit{problem} pada \textit{assignment} yang dipilih.
    
\subsubsection{Hall of Fame}
    \begin{figure}[H]
    	\centering  
    	\includegraphics[scale=0.4]{Page/hall_of_fame.PNG}  
    	\caption{Halaman Hall of Fame}
    	\label{fig:3:hall_of_fame} 
    \end{figure} 
    
    Gambar \ref{fig:3:hall_of_fame} menunjukkan halaman Hall of Fame. Pada halaman ini, terdapat \textit{list} pengguna secara berurutan berdasarkan total nilai yang didapatkannya dari seluruh \textit{assignment}.

\subsubsection{24-Hour Log}
    \begin{figure}[H]
    	\centering  
    	\includegraphics[scale=0.4]{Page/log.PNG}  
    	\caption{Halaman 24-Hour Log}
    	\label{fig:3:log} 
    \end{figure} 
    
    Gambar \ref{fig:3:log} menunjukkan halaman 24-Hour Log. Pada halaman ini, terdapat \textit{list} yang mencatat bila akun yang sama melakukan \textit{login} dengan \textit{IP address} yang berbeda dalam jangka waktu 24 jam. Fitur ini digunakan untuk mendeteksi adanya peminjaman akun.

\subsection{Model, View, Controller}
\label{subs:3:mvc}
SharIF Judge menggunakan \textit{framework} CodeIgniter 3. Seperti yang dibahas pada bagian \ref{subs:2:cimvc}, \textit{framework} CodeIgniter menerapkan pola arsitektur MVC, dengan komponen-komponen \textit{model}, \textit{view}, dan \textit{controller}.

\subsubsection{Model}

Berikut ini adalah \textit{model} pada SharIF Judge:

\begin{itemize}
	\item \verb|Assignment_model| \\ Model untuk menangani tabel \verb|shj_assignments|. Fungsi yang dimiliki:
	\begin{itemize}
		\item \verb|add_assignment($id, $edit)|\\ Menambah atau memperbarui sebuah \textit{assignment}.
		\item \verb|delete_assignment($assignment_id)| \\ Menghapus sebuah \textit{assignment}.
		\item \verb|all_assignments()| \\ Mengambil seluruh \textit{assignment}.
		\item \verb|new_assignment_id()| \\ Menentukan \textit{integer} terkecil yang dapat digunakan sebagai id \textit{assignment} baru.
		\item \verb|all_problems($assignment_id)| \\ Mengambil seluruh \textit{problem} dari \textit{assignment}.
		\item \verb|problem_info($assignment_id, $problem_id)| \\ Mengambil sebuah \textit{problem}.
		\item \verb|assignment_info($assignment_id)| \\ Mengambil sebuah \textit{assignment}.
		\item \verb|is_participant($participants, $username)| \\ Mengembalikan \verb|TRUE| jika \verb|$username| terdapat dalam \verb|$participants|.
		\item \verb|increase_total_submits($assignment_id)| \\ Meningkatkan jumlah total \textit{submit} sebuah \textit{assignment} sebanyak satu.
		\item \verb|set_moss_time($assignment_id)| \\ Memperbarui "\textit{Moss Update Time}" untuk sebuah assignment.
		\item \verb|get_moss_time($assignment_id)| \\ Mengambil "\textit{Moss Update Time}" untuk sebuah assignment.
		\item \verb|save_problem_description($assignment_id, $problem_id, $text, $type)| \\ Menambah atau memperbarui deskripsi sebuah \textit{problem}.
		\item \verb|_update_coefficients($a_id, $extra_time, $finish_time, $new_late_rule)| \\ Memperbarui koefisien seluruh \textit{submission} pada sebuah \textit{assignment}.
	\end{itemize}
	
	\item \verb|Hof_model| \\ Model untuk menangani informasi \textit{hall of fame}. Fungsi yang dimiliki:
	\begin{itemize}
	    \item \verb|get_all_final_submission()| \\ Mengambil seluruh \textit{final submission}.
	    \item \verb|get_all_user_assignments($username)| \\ Mengambil seluruh \textit{assignment} dan \textit{problem} untuk \textit{user} tertentu.
	\end{itemize}
	
	\item \verb|Logs_model| \\ Model untuk menangani tabel \verb|shj_logins|. Fungsi yang dimiliki:
	\begin{itemize}
	    \item \verb|insert_to_logs($username, $ip_adrress)| \\ Menambah sebuah catatan \textit{login} dan menghapus catatan yang sudah melebihi 24 jam.
	    \item \verb|get_all_logs()| \\ Mengambil seluruh catatan \textit{login}.
	\end{itemize}
	
	\item \verb|Notifications_model| \\ Model untuk menangani tabel \verb|shj_notifications|. Fungsi yang dimiliki:
	\begin{itemize}
	    \item \verb|get_all_notifications()| \\ Mengambil seluruh notifikasi.
	    \item \verb|get_latest_notifications()| \\ Mengambil 10 notifikasi terbaru.
	    \item \verb|add_notification($title, $text)| \\ Menambah notifikasi baru.
	    \item \verb|update_notification($id, $title, $text)| \\ Memperbarui sebuah notifikasi. 
	    \item \verb|delete_notification($id)| \\ Menghapus sebuah notifikasi.
	    \item \verb|get_notification($notif_id)| \\ Mengambil sebuah notifikasi.
	    \item \verb|have_new_notification($time)| \\ Mengembalikan \verb|TRUE| jika terdapat notifikasi setelah \verb|$time|.
	\end{itemize}
	
	\item \verb|Queue_model| \\ Model untuk menangani tabel \verb|shj_queue|. Fungsi yang dimiliki:
	\begin{itemize}
	    \item \verb|in_queue($username, $assignment, $problem)| \\ Mengembalikan \verb|TRUE| jika sebuah \textit{submission} sudah berada dalam antrean.
	    \item \verb|get_queue()| \\ Mengambil seluruh antrean.
	    \item \verb|empty_queue()| \\ Mengosongkan antrean.
	    \item \verb|add_to_queue($submit_info)| \\ Menambahkan sebuah \textit{submission} ke dalam antrean.
	    \item \verb|rejudge($assignment_id, $problem_id)| \\ Menambahkan seluruh \textit{submission} dari sebuah \textit{problem} ke dalam antrean untuk dinilai ulang.
	    \item \verb|rejudge_single($submission)| \\  Menambahkan sebuah \textit{submission} ke dalam antrean untuk dinilai ulang.
	    \item \verb|get_first_item()| \\ Mengambil \textit{entry} pertama dari antrean.
	    \item \verb|remove_item($username, $assignment, $problem, $submit_id)| \\ Menghapus sebuah \textit{entry} dari antrean.
	    \item \verb|save_judge_result_in_db ($submission, $type)| \\ Menyimpan hasil penilaian ke dalam \textit{database}.
	\end{itemize}
	
	\item \verb|Scoreboard_model| \\ Model untuk menangani tabel \verb|shj_scoreboard|. Fungsi yang dimiliki:
	\begin{itemize}
	    \item \verb|_generate_scoreboard($assignment_id)| \\ Membuat \textit{scoreboard} untuk sebuah \textit{assignment}.
	    \item \verb|update_scoreboards()| \\ Memperbarui \textit{scoreboard} untuk seluruh \textit{assignment}.
	    \item \verb|update_scoreboard($assignment_id)| \\ Memperbarui \textit{scoreboard} untuk sebuah \textit{assignment}.
	    \item \verb|get_scoreboard($assignment_id)| \\ Mengambil \textit{scoreboard} untuk sebuah \textit{assignment}.
	\end{itemize}
	
	\item \verb|Settings_model| \\ Model untuk menangani tabel \verb|shj_settings|. Fungsi yang dimiliki:
	\begin{itemize}
	    \item \verb|get_setting($key)| \\ Mengambil sebuah pengaturan.
	    \item \verb|set_setting($key, $value)| \\ Memperbarui sebuah pengaturan.
	    \item \verb|get_all_settings()| \\ Mengambil seluruh pengaturan.
	    \item \verb|set_settings($settings)| \\ Memperbarui beberapa pengaturan.
	\end{itemize}
	
	\item \verb|Submit_model| \\ Model untuk menangani tabel \verb|shj_submissions|. Fungsi yang dimiliki:
	\begin{itemize}
	    \item \verb|get_submission($uname, $assignment, $problem, $submit_id)| \\ Mengambil sebuah \textit{submission}.
	    \item \verb|get_final_submissions($a_id, $u_lv, $uname, $p_num, $f_user, $f_prblm)| \\ Mengambil seluruh \textit{final submission} untuk sebuah \textit{assignment}.
	    \item \verb|get_all_submissions($a_id, $u_lv, $uname, $p_num, $f_user, $f_prblm)| \\ Mengambil seluruh \textit{submission} untuk sebuah \textit{assignment}.
	    \item \verb|count_final_submissions($a_id, $u_lv, $uname, $f_user, $f_prblm)| \\ Menghitung jumlah \textit{final submission} dari \textit{user} tertentu.
	    \item \verb|count_all_submissions($a_id, $u_lv, $uname, $f_user, $f_prblm)| \\ Menghitung jumlah \textit{submission} dari \textit{user} tertentu.
	    \item \verb|set_final_submission($uname, $assignment, $problem, $submit_id)| \\ Memperbarui sebuah \textit{submission} menjadi \textit{final}.
	    \item \verb|add_upload_only($submit_info)| \\ Menambahkan hasil dari \textit{submission} \textit{upload only} ke dalam \textit{database}.
	\end{itemize}
	
	\item \verb|User| \\ Model untuk menangani informasi preferensi setiap \textit{user}. Fungsi yang dimiliki:
	\begin{itemize}
	    \item \verb|select_assignment($assignment_id)| \\ Menetapkan \textit{assignment} yang dipilih.
	    \item \verb|save_widget_positions($positions)| \\ Memperbarui posisi \textit{widget}.
	    \item \verb|get_widget_positions()| \\ Mengambil posisi \textit{widget}.
	\end{itemize}
	
	\item \verb|User_model| \\ Model untuk menangani tabel \verb|shj_users|. Fungsi yang dimiliki:
	\begin{itemize}
	    \item \verb|have_user($username)| \\ Mengembalikan \verb|TRUE| jika terdapat \textit{user} dengan nama \verb|$username|.
	    \item \verb|user_id_to_username($user_id)| \\ Mengembalikan \verb|username| dari \textit{user} dengan \verb|id| tertentu.
	    \item \verb|username_to_user_id($username)| \\ Mengembalikan \verb|id| dari \textit{user} dengan \verb|username| tertentu.
	    \item \verb|have_email($email, $username)| \\ Mengembalikan \verb|TRUE| jika terdapat \textit{user} selain \verb|$username| dengan email \verb|$email|.
	    \item \verb|add_user($username, $email, $display_name, $password, $role)| \\ Menambahkan sebuah \textit{user} baru.
	    \item \verb|add_users($text, $send_mail, $delay)| \\ Menambahkan beberapa \textit{user} baru.
	    \item \verb|delete_user($user_id)| \\ Menghapus sebuah \textit{user}.
	    \item \verb|delete_submissions($user_id)| \\ Menghapus seluruh \textit{submission} dari sebuah \textit{user}.
	    \item \verb|validate_user($username, $password)| \\ Mengembalikan \verb|TRUE| jika \verb|$username| dan \verb|$password| valid untuk login.
	    \item \verb|selected_assignment($username)| \\ Mengembalikan \textit{assignment} yang dipilih sebuah \textit{user}.
	    \item \verb|get_names()| \\  Mengembalikan nama dari \textit{user}.
	    \item \verb|update_profile($user_id)| \\ Memperbarui sebuah \textit{user}.
	    \item \verb|send_password_reset_mail($email)| \\ Mengirim \textit{email} untuk \textit{reset password}.
	    \item \verb|passchange_is_valid($passchange_key)| \\ Mengembalikan \verb|TRUE| jika kunci untuk \textit{reset password} valid.
	    \item \verb|reset_password($passchange_key, $newpassword)| \\ Memperbarui \textit{password} menjadi kunci \textit{reset password}.
	    \item \verb|get_all_users()| \\ Mengambil seluruh \textit{user}.
	    \item \verb|get_user($user_id)| \\ Mengambil sebuah \textit{user}.
	    \item \verb|update_login_time($username)| \\ Memperbarui catatan \textit{login} sebuah \textit{user}.
	\end{itemize}
\end{itemize}

\subsubsection{View}

\textit{View} pada SharIF Judge terbagi menjadi beberapa folder:

\begin{itemize}
    \begin{figure}[H]
    	\centering  
    	\includegraphics[scale=0.4]{View/sidebar}  
    	\caption{\texttt{side\_bar.twig}}
    	\label{fig:3:viewsidebar} 
    \end{figure} 
    \item \verb|templates| \\ Menyimpan komponen dasar halaman. Gambar \ref{fig:3:viewsidebar} adalah salah satu contoh komponen dasar halaman.
    
    \begin{figure}[H]
    	\centering  
    	\includegraphics[scale=0.4]{View/submit}  
    	\caption{\texttt{submit.twig}}
    	\label{fig:3:viewsubmit} 
    \end{figure} 
    \item \verb|pages| \\ Menyimpan komponen utama halaman. Gambar \ref{fig:3:viewsubmit} adalah salah satu contoh komponen utama halaman.
    
    \begin{figure}[H]
    	\centering  
    	\includegraphics[scale=0.4]{View/404}  
    	\caption{\texttt{error\_404.php}}
    	\label{fig:3:view404} 
    \end{figure} 
	\item \verb|errors| \\ Menyimpan tampilan halaman error. Gambar \ref{fig:3:view404} adalah salah satu contoh halaman error.
\end{itemize}

\subsubsection{Controller}

Berikut ini adalah \textit{controller} pada SharIF Judge:

\begin{itemize}
	\item \verb|Assignments| \\ Controller untuk menangani \textit{assignments}. Fungsi yang dimiliki:
	\begin{itemize}
		\item \verb|select()| \\ Memilih \textit{assignment} yang sedang ditampilkan.
		\item \verb|pdf($assignment_id, $problem_id)| \\ Mengunduh \textit{file} PDF dari sebuah \textit{assignment}.
		\item \verb|downloadtestsdesc($assignment_id)| \\ Mengunduh \textit{file} \textit{test case} dari sebuah \textit{assignment}.
		\item \verb|download_submissions($type, $assignment_id)| \\ Mengunduh seluruh \textit{file} \textit{final submission} dari sebuah \textit{assignment}.
		\item \verb|delete($assignment_id)| \\ Menghapus sebuah \textit{assignment}.
		\item \verb|add()| \\ Menambah atau memperbarui \textit{assignment}.
		\item \verb|edit($assignment_id)| \\ Memperbarui \textit{assignment}.
	\end{itemize}
	
	\item \verb|Dashboard| \\Controller untuk menangani halaman \textit{Dashboard}. Fungsi yang dimiliki:
	\begin{itemize}
		\item \verb|widget_positions()| \\ Menyimpan posisi \textit{widget} dari \textit{user}.
	\end{itemize}
	
	\item \verb|Halloffame| \\ Controller untuk menangani halaman \textit{Hall of Fame} . Fungsi yang dimiliki:
	\begin{itemize}
		\item \verb|hof_details()| \\ Mengambil data yang diperlukan untuk \textit{hall of fame}.
	\end{itemize}
	
	\item \verb|Install| \\ Controller untuk menangani instalasi SharIF Judge.
	
	\item \verb|Login| \\ Controller untuk menangani halaman-halaman \textit{login}. Fungsi yang dimiliki:
	\begin{itemize}
		\item \verb|register()| \\ Registrasi \textit{user} baru dan menampilkan halaman \textit{register}.
		\item \verb|logout()| \\ \textit{Log out} user saat ini dan mengalihkan ke halaman \textit{login}.
		\item \verb|lost()| \\ Menangani email dan menampilkan halaman untuk meminta \textit{reset password}.
		\item \verb|reset($passchange_key)| \\ Memproses dan menampilkan halaman untuk ubah \textit{reset password}.
	\end{itemize}
	
	\item \verb|Logs| \\ Controller untuk menangani halaman \textit{24-hour Log}.
	\begin{itemize}
        \item \verb|index()| Mengambil data yang diperlukan dan menampilkan halaman \textit{24-hour Log}.
	\end{itemize}
	
	\item \verb|Moss| \\ Controller untuk menangani halaman \textit{Detect Similar Codes} . Fungsi yang dimiliki:
	\begin{itemize}
		\item \verb|update($assignment_id)| \\ Memperbarui informasi pada halaman \textit{Detect Similar Codes}.
		\item \verb|_detect($assignment_id)| \\ Menjalankan Moss untuk mendeteksi kesamaan kode.
	\end{itemize}
	
	\item \verb|Notifications| \\ Controller untuk menangani halaman \textit{Notifications}. Fungsi yang dimiliki:
	\begin{itemize}
		\item \verb|add()| \\ Menambahkan notifikasi baru dan menampilkan halaman \textit{New Notification}.
		\item \verb|edit($notif_id)| \\ Memperbarui sebuah notifikasi.
		\item \verb|delete()| \\ Menghapus sebuah notifikasi.
		\item \verb|check()| \\ Memeriksa adanya notifikasi baru.
	\end{itemize}
	
	\item \verb|Problems| \\ Controller untuk menangani halaman \textit{Problems}. Fungsi yang dimiliki:
	\begin{itemize}
		\item \verb|index($assignment_id, $problem_id = 1)| \\ Mengambil data yang diperlukan dan menampilkan halaman \textit{Problems}.
		\item \verb|edit($type = 'md', $assignment_id, $problem_id = 1)| \\ Memperbarui deskripsi \textit{problem} dan menampilkan halaman \textit{Edit Problem Description}.
	\end{itemize}
	
	\item \verb|Profile| \\ Controller untuk menangani halaman \textit{Profile}. Fungsi yang dimiliki:
	\begin{itemize}
		\item \verb|index($user_id)| \\ Mengambil data yang diperlukan dan menampilkan halaman \textit{Profile}.
		\item \verb|_password_check($str)| \\ Memeriksa apakah \textit{password} sesuai dengan syarat.
		\item \verb|_password_again_check($str)| \\ Memeriksa apakah \textit{password again} sama dengan \textit{password} yang dimasukkan.
		\item \verb|_email_check($email)| \\ Memeriksa apakah terdapat user dengan alamat email tertentu.
		\item \verb|_role_check($role)| \\ Memeriksa \textit{role} yang dimiliki \textit{user}.
	\end{itemize}
	
	\item \verb|Queue| \\ Controller untuk menangani halaman \textit{Queue}. Fungsi yang dimiliki:
	\begin{itemize}
        \item \verb|index()| \\ Mengambil data yang diperlukan dan menampilkan halaman \textit{Queue}.
        \item \verb|pause()| \\ Memberhentikan antrean.
        \item \verb|resume()| \\ Melanjutkan antrean.
        \item \verb|empty_queue()| \\ Mengosongkan antrean.
	\end{itemize}
	
	\item \verb|Queueprocess| \\Controller untuk menangani proses penilaian kode. Fungsi yang dimiliki:
	\begin{itemize}
        \item \verb|run()| \\ Menilai kode satu per satu dari antrean.
	\end{itemize}
	
	\item \verb|Rejudge| \\ Controller untuk menangani halaman \textit{Rejudge}. Fungsi yang dimiliki:
	\begin{itemize}
        \item \verb|index()| \\ Mengambil data yang diperlukan dan menampilkan halaman \textit{Rejudge}.
        \item \verb|rejudge_single()| \\ Melakukan penilaian ulang untuk sebuah \textit{submission}.
	\end{itemize}
	
	\item \verb|Server_time| \\ Controller untuk menangani sinkronisasi waktu server. Fungsi yang dimiliki:
	\begin{itemize}
        \item \verb|index()| \\ Mengembalikan waktu server.
	\end{itemize}
	
	\item \verb|Submissions| \\ Controller untuk menangani unduh \textit{submissions} menjadi file Excel. Fungsi yang dimiliki:
	\begin{itemize}
        \item \verb|_download_excel($view)| \\ Menggunakan \textit{library} PHPExcel untuk membuat file excel.
        \item \verb|final_excel()| \\ Mengunduh data \textit{final submissions} sebagai file excel.
        \item \verb|all_excel()| \\ Mengunduh data \textit{final submissions} sebagai file excel.
        \item \verb|the_final()| \\ Mengambil dan menampilkan data \textit{final submissions} yang akan diunduh.
        \item \verb|all()| \\ Mengambil dan menampilkan data \textit{submissions} yang akan diunduh.
        \item \verb|select()| \\ Memilih \textit{final submission}.
        \item \verb|view_code()| \\ Menampilkan kode, \textit{result}, atau \textit{log} dari \textit{submission}.
        \item \verb|download_file()| \\ Mengunduh file excel.
	\end{itemize}
	
	\item \verb|Submit| \\ Controller untuk menangani \textit{submissions}. Fungsi yang dimiliki:
	\begin{itemize}
        \item \verb|_language_to_type($language)| \\ Mengembalikan kode singkatan dari bahasa pemrograman.
        \item \verb|_match($type, $extension)| \\ Memeriksa apakah bahasa pemrograman dan tipe file sesuai.
        \item \verb|_check_language($str)| \\ Memeriksa apakah bahasa pemrograman yang dipilih valid.
        \item \verb|index()| \\ Mengambil data yang diperlukan dan menampilkan halaman \textit{Submit}.
        \item \verb|_upload()| \\ Menyimpan file yang diunggah dan menambahkannya ke dalam antrean.
	\end{itemize}
	
	\item \verb|Users| \\ Controller untuk menangani halaman \textit{Users}. Fungsi yang dimiliki:
	\begin{itemize}
        \item \verb|index()| \\ Mengambil data yang diperlukan dan menampilkan halaman \textit{Users}.
        \item \verb|add()| \\ Menambah \textit{user} baru dan menampilkan halaman \textit{Add Users}.
        \item \verb|delete()| \\ Menghapus \textit{user}.
        \item \verb|delete_submissions()| \\ Menghapus seluruh \textit{submission} dari sebuah \textit{user}.
         \item \verb|list_excel()| \\ Menggunakan \textit{library} PHPExcel untuk membuat file excel dari \textit{list user}.
	\end{itemize}
\end{itemize}

\subsection{Antrean Penilaian Kode}
\label{subsec:3:antrean} 
Pada SharIF Judge, seluruh kode yang dikumpulkan pengguna akan dijalankan satu per satu dalam antrean untuk dinilai. Tahap-tahap yang dilalui sebuah kode hingga penilaian selesai adalah sebagai berikut:

\begin{itemize}
    \item \textit{Controller} \verb|Submit| menyimpan \textit{file} kode pada folder sesuai dengan \textit{assignment} dan \textit{problem} yang dipilih.
    \item \textit{Model} \verb|Queue_model| menyimpan alamat kode sebagai sebuah \textit{entry} \textit{submission} di tabel \verb|submission|, lalu kode dimasukkan dalam antrean sebagai sebuah \textit{entry} di tabel \verb|queue|.
    \item \textit{Controller} \verb|Queueprocess| membaca \textit{entry} tabel \verb|queue| satu per satu untuk dinilai dengan menjalankan \verb|tester.sh|.
    \item \verb|tester.sh| mengompilasi kode, menjalankan kode dengan tes kasus, menilai hasilnya dengan kunci jawaban, lalu mengembalikan hasil penilaian.
    \item \textit{Controller} \verb|Queueprocess| menyimpan nilai kembalian pada database \verb|submissions| dan menghapus \textit{entry} dari tabel \verb|queue|.
\end{itemize}

\section{Analisis Sistem Usulan}
\label{sec:3:analisisusulan} 
Fitur yang akan diimplementasikan pada SharIF Judge adalah sebagai berikut:

\begin{itemize}
    \item Melihat soal \\ Saat ini SharIF Judge memiliki kemampuan untuk menyimpan soal dalam bentuk PDF. Soal tersebut akan ditampilkan secara langsung pada \textit{browser}.
    \item Mengetik kode \\ Editor teks yang memiliki kemampuan untuk membantu pembuatan kode, seperti \textit{syntax highlighting}.
    \item Menyimpan kode \\ Kemampuan untuk menyimpan dan memuat kembali kode yang sudah dibuat pada server.
    \item Menjalankan kode dengan tes kasus \\  Kode yang sudah dibuat dapat dijalankan dengan tes kasus yang disediakan oleh pengguna.
    \item Mengumpulkan Kode dari Editor \\  Melakukan \textit{submit} kode yang sudah dibuat pada editor.
\end{itemize}
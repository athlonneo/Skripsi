\chapter{Implementasi dan Pengujian}
\label{chap:implementasidanpengujian}

Bab ini membahas mengenai implementasi dan pengujian perangkat lunak SharIF Judge.

\section{Lingkungan Implementasi dan Pengujian}
\label{sec:5:lingkungan}

Implementasi perangkat lunak dilakukan pada perangkat penulis dengan spesifikasi sebagai berikut:

\begin{itemize}
    \item Perangkat Keras:
    \begin{itemize}
        \item \textit{Processor}: Intel Core i5-7600
        \item \textit{Random Access Memory}: 16GB DDR4
        \item \textit{Graphics Processing Unit}: Nvidia GeForce GTX 1070
        \item \textit{Storage}: 500GB SSD dan 2TB HDD
    \end{itemize}
        \item Perangkat Lunak:
    \begin{itemize}
        \item \textit{Operating System}: Windows 10 Home 64-bit
        \item \textit{Windows Subsystem for Linux}: Ubuntu 20.04.2 LTS
    \end{itemize}
\end{itemize}

\section{Implementasi}
\label{sec:5:implementasi}

\section{Pengujian}
\label{sec:5:pengujian}

\subsection{Pengujian Fungsional}
\label{subsec:5:fungsional}

Pengujian fungsional dilakukan secara lokal pada perangkat penulis. Berikut ini pengujian yang dilakukan terhadap fitur-fitur yang sudah diimplementasi:

\begin{itemize}
    \item Menampilkan soal
    \begin{itemize}
        \item Hasil yang diharapkan: Soal pdf ditampilkan di halaman Submit
        \item Hasil perangkat lunak: Sesuai
    \end{itemize}
\end{itemize}

\subsection{Pengujian Eksperimental}
\label{subsec:5:eksperimental}

Pengujian eksperimental dilakukan pada mata kuliah Dasar-dasar Pemrograman semester 51 Teknik Informatika Unpar. Perangkat lunak diuji pada \textit{judge} dengan alamat \verb|http://daspro.labftis.net|. Seluruh persoalan dan masukan yang diterima selama mata kuliah Dasar-dasar Pemrograman dicatat pada \verb|https://github.com/athlonneo/SharIF-Judge/issues|.

\begin{itemize}
    \item Tampilan UI
\end{itemize}



\chapter{Kesimpulan dan Saran}
\label{chap:kesimpulandansaran}

\section{Kesimpulan}
\label{sec:6:kesimpulan}

Berdasarkan hasil analisis, implementasi, dan pengujian fitur pada perangkat lunak SharIF Judge yang dikembangkan, diperoleh kesimpulan sebagai berikut:

\begin{itemize}
    \item Dengan memanfaatkan fitur yang sudah ada dan menambahkan fitur baru, halaman Submit pada SharIF Judge dapat berfungsi sebagai {\it Integrated Development Environment} (IDE), dengan kemampuan untuk mengedit, mengompilasi, menjalankan, dan menguji kode program. Berikut ini adalah fitur-fitur yang diimplementasikan:
    \begin{itemize}
        \item Menampilkan soal
        \item Mengedit kode
        \item Menyimpan dan memuat kode
        \item Menjalankan kode dengan tes kasus
        \item Mengumpulkan kode melalui IDE
    \end{itemize}
    
    \item Seluruh masukan dan penyelesaian masalah yang ditemukan saat pengujian di mata kuliah Dasar-dasar Pemrograman sudah diimplementasikan, termasuk dengan salah satu penyelesaian masalah yang berasal dari SharIF Judge versi sebelumnya dan tidak berhubungan dengan skripsi ini.
    
    \item Melalui hasil survei, dapat disimpulkan bahwa fitur-fitur yang diimplementasikan pada SharIF Judge sudah cukup nyaman untuk digunakan. Namun, terdapat sebagian besar mahasiswa yang belum mencoba beberapa fitur.
\end{itemize}

\section{Saran}
\label{sec:6:saran}

Berdasarkan hasil pengembangan yang dilakukan, berikut adalah saran-saran untuk pengembangan selanjutnya:

\begin{itemize}
    \item Mengimplementasikan fitur merekam ketikan dengan memanfaatkan Ace, untuk membantu pengawasan terhadap mahasiswa selama kegiatan kuliah.
    \item Menguji perangkat lunak pada mata kuliah pemrograman lainnya, untuk mendapatkan umpan balik dari mahasiswa yang sudah berpengalaman dalam menggunakan IDE lain.
\end{itemize}
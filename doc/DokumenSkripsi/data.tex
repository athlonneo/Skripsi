%_____________________________________________________________________________
%=============================================================================
% data.tex v11 (24-07-2020) dibuat oleh Lionov - Informatika FTIS UNPAR
%
% Perubahan pada versi 11 (24-07-2020)
%	- Penambahan enumitem dan nosep untuk semua list, untuk menghemat kertas
%   - Bagian V: penambahan opsi Daftar Kode Program dan Daftar Notasi
%   - Bagian XIV: menjadi Bagian XVI
%   - Bagian XV: menjadi Bagian XVII
%   - Bagian XIV yang baru: untuk pilihan jenis tanda tangan mahasiswa
%   - Bagian XV yang baru: untuk pilihan munculnya tanda tangan digital untuk
%     dosen/pejabat
%_____________________________________________________________________________
%=============================================================================

%=============================================================================
% 								PETUNJUK
%=============================================================================
% Ini adalah file data (data.tex)
% Masukkan ke dalam file ini, data-data yang diperlukan oleh template ini
% Cara memasukkan data dijelaskan di setiap bagian
% Data yang WAJIB dan HARUS diisi dengan baik dan benar adalah SELURUHNYA !!
% Hilangkan tanda << dan >> jika anda menemukannya
%=============================================================================

%_____________________________________________________________________________
%=============================================================================
% 								BAGIAN 0
%=============================================================================
% Entri untuk memperbaiki posisi "DAFTAR ISI" jika tidak berada di bagian 
% tengah halaman. Sayangnya setiap sistem menghasilkan posisi yang berbeda.
% Isilah dengan 0 atau 1 (e.g. \daftarIsiError{1}). 
% Pemilihan 0 atau 1 silahkan disesuaikan dengan hasil PDF yang dihasilkan.
%=============================================================================
\daftarIsiError{0}   
%\daftarIsiError{1}   
%=============================================================================

%_____________________________________________________________________________
%=============================================================================
% 								BAGIAN I
%=============================================================================
% Tambahkan package2 lain yang anda butuhkan di sini
%=============================================================================
\usepackage{booktabs} 
\usepackage{longtable}
\usepackage{amssymb}
\usepackage{todo}
\usepackage{verbatim} 		%multiline comment
\usepackage{pgfplots}
\usepackage{enumitem}
%\overfullrule=3mm % memperlihatkan overfull 

\definecolor{diffstart}{RGB}{169,169,169}
\definecolor{diffincl}{RGB}{0,128,0}
\definecolor{diffrem}{RGB}{255,0,0}

\lstdefinelanguage{diff}{
	morecomment=[f][\color{diffstart}]{@@},
	morecomment=[f][\color{diffrem}]{- },
	morecomment=[f][\color{diffincl}]{+ },
}

\lstdefinelanguage{JavaScript}{
  morekeywords=[1]{break, continue, delete, else, for, function, if, in,
    new, return, this, typeof, var, void, while, with},
  % Literals, primitive types, and reference types.
  morekeywords=[2]{false, null, true, boolean, number, undefined,
    Array, Boolean, Date, Math, Number, String, Object},
  % Built-ins.
  morekeywords=[3]{eval, parseInt, parseFloat, escape, unescape},
  sensitive,
  morecomment=[s]{/*}{*/},
  morecomment=[l]//,
  morecomment=[s]{/**}{*/}, % JavaDoc style comments
  morestring=[b]',
  morestring=[b]"
}[keywords, comments, strings]

\lstdefinelanguage{CSS}{
  morekeywords={accelerator,azimuth,background,background-attachment,
    background-color,background-image,background-position,
    background-position-x,background-position-y,background-repeat,
    behavior,border,border-bottom,border-bottom-color,
    border-bottom-style,border-bottom-width,border-collapse,
    border-color,border-left,border-left-color,border-left-style,
    border-left-width,border-right,border-right-color,
    border-right-style,border-right-width,border-spacing,
    border-style,border-top,border-top-color,border-top-style,
    border-top-width,border-width,bottom,caption-side,clear,
    clip,color,content,counter-increment,counter-reset,cue,
    cue-after,cue-before,cursor,direction,display,elevation,
    empty-cells,filter,float,font,font-family,font-size,
    font-size-adjust,font-stretch,font-style,font-variant,
    font-weight,height,ime-mode,include-source,
    layer-background-color,layer-background-image,layout-flow,
    layout-grid,layout-grid-char,layout-grid-char-spacing,
    layout-grid-line,layout-grid-mode,layout-grid-type,left,
    letter-spacing,line-break,line-height,list-style,
    list-style-image,list-style-position,list-style-type,margin,
    margin-bottom,margin-left,margin-right,margin-top,
    marker-offset,marks,max-height,max-width,min-height,
    min-width,-moz-binding,-moz-border-radius,
    -moz-border-radius-topleft,-moz-border-radius-topright,
    -moz-border-radius-bottomright,-moz-border-radius-bottomleft,
    -moz-border-top-colors,-moz-border-right-colors,
    -moz-border-bottom-colors,-moz-border-left-colors,-moz-opacity,
    -moz-outline,-moz-outline-color,-moz-outline-style,
    -moz-outline-width,-moz-user-focus,-moz-user-input,
    -moz-user-modify,-moz-user-select,orphans,outline,
    outline-color,outline-style,outline-width,overflow,
    overflow-X,overflow-Y,padding,padding-bottom,padding-left,
    padding-right,padding-top,page,page-break-after,
    page-break-before,page-break-inside,pause,pause-after,
    pause-before,pitch,pitch-range,play-during,position,quotes,
    -replace,richness,right,ruby-align,ruby-overhang,
    ruby-position,-set-link-source,size,speak,speak-header,
    speak-numeral,speak-punctuation,speech-rate,stress,
    scrollbar-arrow-color,scrollbar-base-color,
    scrollbar-dark-shadow-color,scrollbar-face-color,
    scrollbar-highlight-color,scrollbar-shadow-color,
    scrollbar-3d-light-color,scrollbar-track-color,table-layout,
    text-align,text-align-last,text-decoration,text-indent,
    text-justify,text-overflow,text-shadow,text-transform,
    text-autospace,text-kashida-space,text-underline-position,top,
    unicode-bidi,-use-link-source,vertical-align,visibility,
    voice-family,volume,white-space,widows,width,word-break,
    word-spacing,word-wrap,writing-mode,z-index,zoom},
  morestring=[s]{:}{;},
  sensitive,
  morecomment=[s]{/*}{*/}
}

%=============================================================================

%_____________________________________________________________________________
%=============================================================================
% 								BAGIAN II
%=============================================================================
% Mode dokumen: menentukan halaman depan dari dokumen, apakah harus mengandung 
% prakata/pernyataan/abstrak dll (termasuk daftar gambar/tabel/isi) ?
% - final 		: hanya untuk buku skripsi, dicetak lengkap: judul ina/eng, 
%   			  pengesahan, pernyataan, abstrak ina/eng, untuk, kata 
%				  pengantar, daftar isi (daftar tabel dan gambar tetap 
%				  opsional dan dapat diatur), seluruh bab dan lampiran.
%				  Otomatis tidak ada nomor baris dan singlespacing
% - sidangakhir	: buku sidang akhir = buku final - (pengesahan + pernyataan +
%   			  untuk + kata pengantar)
%				  Otomatis ada nomor baris dan onehalfspacing 
% - sidang 		: untuk sidang 1, buku sidang = buku sidang akhir - (judul 
%				  eng + abstrak ina/eng)
%				  Otomatis ada nomor baris dan onehalfspacing
% - bimbingan	: untuk keperluan bimbingan, hanya terdapat bab dan lampiran
%   			  saja, bab dan lampiran yang hendak dicetak dapat ditentukan 
%				  sendiri (nomor baris dan spacing dapat diatur sendiri)
% Mode default adalah 'template' yang menghasilkan isian berwarna merah, 
% aktifkan salah satu mode di bawah ini :
%=============================================================================
%\mode{bimbingan} 		% untuk keperluan bimbingan
%\mode{sidang} 			% untuk sidang 1
%\mode{sidangakhir} 	% untuk sidang 2 / sidang pada Skripsi 2(IF)
%\mode{final} 			% untuk mencetak buku skripsi 
%=============================================================================
\mode{sidangakhir}

%_____________________________________________________________________________
%=============================================================================
% 								BAGIAN III
%=============================================================================
% Line numbering: penomoran setiap baris, nomor baris otomatis di-reset ke 1
% setiap berganti halaman.
% Sudah dikonfigurasi otomatis untuk mode final (tidak ada), mode sidang (ada)
% dan mode sidangakhir (ada).
% Untuk mode bimbingan, defaultnya ada (\linenumber{yes}), jika ingin 
% dihilangkan, isi dengan "no" (i.e.: \linenumber{no})
% Catatan:
% - jika nomor baris tidak kembali ke 1 di halaman berikutnya, compile kembali
%   dokumen latex anda
% - bagian ini hanya bisa diatur di mode bimbingan
%=============================================================================
%\linenumber{no} 
\linenumber{yes}
%=============================================================================

%_____________________________________________________________________________
%=============================================================================
% 								BAGIAN IV
%=============================================================================
% Linespacing: jarak antara baris 
% - single	: otomatis jika ingin mencetak buku skripsi, opsi yang 
%			     disediakan untuk bimbingan, jika pembimbing tidak keberatan 
%			     (untuk menghemat kertas)
% - onehalf	: otomatis jika ingin mencetak dokumen untuk sidang
% - double 	: jarak yang lebih lebar lagi, jika pembimbing berniat memberi 
%             catatan yg banyak di antara baris (dianjurkan untuk bimbingan)
% Catatan: bagian ini hanya bisa diatur di mode bimbingan
%=============================================================================
\linespacing{single}
%\linespacing{onehalf}
%\linespacing{double}
%=============================================================================

%_____________________________________________________________________________
%=============================================================================
% 								BAGIAN V
%=============================================================================
% Tidak semua skripsi memuat gambar, tabel, kode program, dan/atau notasi. 
% Untuk skripsi yang tidak memuat hal-hal tersebut, maka tidak diperlukan 
% Daftar Gambar, Daftar Tabel, Daftar Kode Program, dan/atau Daftar Notasi. 
% Sayangnya hal tsb sulit dilakukan secara manual karena membutuhkan 
% kedisiplinan pengguna template.  
% Jika tidak ingin menampilkan satu/lebih daftar-daftar tersebut (misalnya 
% untuk bimbingan), isi dengan "no" (e.g. \gambar{no})
%=============================================================================
\gambar{yes}
%\gambar{no}
%\tabel{yes}
\tabel{no} 
%\kode{yes}
\kode{no} 
%\notasi{yes}
\notasi{no}
%=============================================================================

%_____________________________________________________________________________
%=============================================================================
% 								BAGIAN VI
%=============================================================================
% Pada mode final, sidang da sidangkahir, seluruh bab yang ada di folder "Bab"
% dengan nama file bab1.tex, bab2.tex s.d. bab9.tex akan dicetak terurut, 
% apapun isi dari perintah \bab.
% Pada mode bimbingan, jika ingin:
% - mencetak seluruh bab, isi dengan 'all' (i.e. \bab{all})
% - mencetak beberapa bab saja, isi dengan angka, pisahkan dengan ',' 
%   dan bab akan dicetak terurut sesuai urutan penulisan (e.g. \bab{1,3,2}). 
% Catatan: Jika ingin menambahkan bab ke-3 s.d. ke-9, tambahkan file 
% bab3.tex, bab4.tex, dst di folder "Bab". Untuk bab ke-10 dan 
% seterusnya, harus dilakukan secara manual dengan mengubah file skripsi.tex
% Catatan: bagian ini hanya bisa diatur di mode bimbingan
%=============================================================================
\bab{all}
%=============================================================================

%_____________________________________________________________________________
%=============================================================================
% 								BAGIAN VII
%=============================================================================
% Pada mode final, sidang dan sidangkhir, seluruh lampiran yang ada di folder 
% "Lampiran" dengan nama file lampA.tex, lampB.tex s.d. lampJ.tex akan dicetak 
% terurut, apapun isi dari perintah \lampiran.
% Pada mode bimbingan, jika ingin:
% - mencetak seluruh lampiran, isi dengan 'all' (i.e. \lampiran{all})
% - mencetak beberapa lampiran saja, isi dengan huruf, pisahkan dengan ',' 
%   dan lampiran akan dicetak terurut sesuai urutan (e.g. \lampiran{A,E,C}). 
% - tidak mencetak lampiran apapun, isi dengan "none" (i.e. \lampiran{none})
% Catatan: Jika ingin menambahkan lampiran ke-C s.d. ke-I, tambahkan file 
% lampC.tex, lampD.tex, dst di folder Lampiran. Untuk lampiran ke-J dan 
% seterusnya, harus dilakukan secara manual dengan mengubah file skripsi.tex
% Catatan: bagian ini hanya bisa diatur di mode bimbingan
%=============================================================================
\lampiran{all}
%=============================================================================

%_____________________________________________________________________________
%=============================================================================
% 								BAGIAN VIII
%=============================================================================
% Data diri dan skripsi/tugas akhir
% - namanpm		: Nama dan NPM anda, penggunaan huruf besar untuk nama harus 
%				  benar dan gunakan 10 digit npm UNPAR, PASTIKAN BAHWA 
%				  BENAR !!! (e.g. \namanpm{Jane Doe}{1992710001}
% - judul 		: Dalam bahasa Indonesia, perhatikan penggunaan huruf besar, 
%				  judul tidak menggunakan huruf besar seluruhnya !!! 
% - tanggal 	: isi dengan {tangga}{bulan}{tahun} dalam angka numerik, 
%				  jangan menuliskan kata (e.g. AGUSTUS) dalam isian bulan.
%			  	  Tanggal ini adalah tanggal dimana anda akan melaksanakan 
%				  sidang ujian akhir skripsi/tugas akhir
% - pembimbing	: pembimbing anda, lihat daftar dosen di file dosen.tex
%				  jika pembimbing hanya 1, kosongkan parameter kedua 
%				  (e.g. \pembimbing{\JND}{} ), \JND adalah kode dosen
% - penguji 	: para penguji anda, lihat daftar dosen di file dosen.tex
%				  (e.g. \penguji{\JHD}{\JCD} )
% !!Lihat singkatan pembimbing dan penguji anda di file dosen.tex!!
% Petunjuk: hilangkan tanda << & >>, dan isi sesuai dengan data anda
%=============================================================================
\namanpm{Nicholas Aditya Halim}{2017730018}
\tanggal{<<tanggal>>}{<<bulan>>}{2022}         %isi bulan dengan angka
\pembimbing{\PAN}{}    
\penguji{<<penguji 1>>}{<<penguji 2>>} 
%=============================================================================

%_____________________________________________________________________________
%=============================================================================
% 								BAGIAN IX
%=============================================================================
% Judul dan title : judul bhs indonesia dan inggris
% - judulINA: judul dalam bahasa indonesia
% - judulENG: title in english
% Petunjuk: 
% - hilangkan tanda << & >>, dan isi sesuai dengan data anda
% - langsung mulai setelah '{' awal, jangan mulai menulis di baris bawahnya
% - gunakan \texorpdfstring{\\}{} untuk pindah ke baris baru
% - judul TIDAK ditulis dengan menggunakan huruf besar seluruhnya !!
%=============================================================================
\judulINA{Implementasi Editor Kode pada SharIF Judge}
\judulENG{Code Editor Implementation on SharIF Judge}
%_____________________________________________________________________________
%=============================================================================
% 								BAGIAN X
%=============================================================================
% Abstrak dan abstract : abstrak bhs indonesia dan inggris
% - abstrakINA: abstrak bahasa indonesia
% - abstrakENG: abstract in english 
% Petunjuk: 
% - hilangkan tanda << & >>, dan isi sesuai dengan data anda
% - langsung mulai setelah '{' awal, jangan mulai menulis di baris bawahnya
%=============================================================================
\abstrakINA{
SharIF Judge adalah sebuah \textit{online judge} (sebuah sistem \textit{online} yang berfungsi untuk mengevaluasi kode program) untuk bahasa pemrograman C, C++, Java dan Python yang dibangun menggunakan CodeIgniter dan Bash. SharIF Judge digunakan pada beberapa mata kuliah pemrograman Teknik Informatika Unpar untuk mempermudah proses pengumpulan dan penilaian kode program.
Situasi pandemi Covid-19 menyebabkan seluruh kegiatan kuliah dilaksanakan secara \textit{online}. Pada umumnya, kegiatan praktikum dan ujian Pada umumnya, kegiatan praktikum dan ujian pada mata kuliah pemrograman Teknik Informatika Unpar dapat diawasi secara langsung oleh dosen dan asisten dosen di lab komputer.  Namun, pengawasan menjadi lebih sulit untuk dilakukan saat kuliah dilaksanakan secara \textit{online}.
{\it Integrated Development Environment} (IDE) akan diimplementasikan pada SharIF Judge, dengan kemampuan untuk memfasilitasi proses penulisan kode, lalu mengompilasi, menjalankan, dan mengujinya. Sebagai sebuah IDE, selanjutnya dapat ditambahkan fitur yang dapat membantu pengawasan terhadap mahasiswa selama kegiatan kuliah, seperti merekam ketikan dan mendeteksi ketika mahasiswa membuka \textit{tab} atau aplikasi lain.
Fitur melihat soal, mengetik, menyimpan, menjalankan, dan mengumpulkan kode melalui IDE diimplementasikan dan diuji pada mata kuliah Dasar-dasar Pemrograman semester ganjil 2021/2022 Teknik Informatika Unpar. Berdasarkan hasil pengujian, seluruh masalah yang ditemukan berhasil diperbaiki, dan seluruh fitur yang diimplementasikan sudah berfungsi dengan baik.
}
\abstrakENG{
SharIF Judge is an online judge (an online system that serves to evaluate program code) for C, C++, Java and Python built using CodeIgniter and Bash. SharIF Judge is used on several programming courses in Unpar Informatics Engineering Study Program to help with code submission and scoring.
The Covid-19 Pandemic caused every learning activities to be done online. Usually, practical lectures and exams in Unpar Informatics Engineering can be supervised directly by lecturers and assistants in the computer lab. However, supervision becomes more difficult to do when lectures are carried out online.
Integrated Development Environment (IDE) will be implemented on SharIF Judge with the ability to facilitate the process of code writing, then compile, run, and test it. As an IDE, further features can be added to help supervise students during learning activities, such as recording typing activities and detecting application inactivity.
Features to show problems, type, save, run, and submit code from IDE is implemented and tested on odd semester 2021/2022 \textit{Dasar-dasar Pemrograman} (Basic Programming) course in Unpar Informatics Engineering. Based on the results, every problem encountered has been fixed successfully and all implemented features have performed adequately.

} 
%=============================================================================

%_____________________________________________________________________________
%=============================================================================
% 								BAGIAN XI
%=============================================================================
% Kata-kata kunci dan keywords : diletakkan di bawah abstrak (ina dan eng)
% - kunciINA: kata-kata kunci dalam bahasa indonesia
% - kunciENG: keywords in english
% Petunjuk: hilangkan tanda << & >>, dan isi sesuai dengan data anda.
%=============================================================================
\kunciINA{\textit{Online judge}, \textit{Integrated Development Environment}}
\kunciENG{Online judge, Integrated Development Environment}
%=============================================================================

%_____________________________________________________________________________
%=============================================================================
% 								BAGIAN XII
%=============================================================================
% Persembahan : kepada siapa anda mempersembahkan skripsi ini ...
% Petunjuk: hilangkan tanda << & >>, dan isi sesuai dengan data anda.
%=============================================================================
\untuk{<<kepada siapa anda mempersembahkan skripsi ini\ldots?>>}
%=============================================================================

%_____________________________________________________________________________
%=============================================================================
% 								BAGIAN XIII
%=============================================================================
% Kata Pengantar: tempat anda menuliskan kata pengantar dan ucapan terima 
% kasih kepada yang telah membantu anda bla bla bla ....  
% Petunjuk: hilangkan tanda << & >>, dan isi sesuai dengan data anda.
%=============================================================================
\prakata{<<Tuliskan kata pengantar dari anda di sini \ldots>>} 
%=============================================================================

%_____________________________________________________________________________
%=============================================================================
% 								BAGIAN XIV
%=============================================================================
% Jenis tandatangan di lembar pernyataan mahasiswa tentang plagiarisme.
% Ada 4 pilihan:
%   - digital   : diisi menggunakan digital signature (menggunakan pengolah
%                 pdf seperti Adobe Acrobat Reader DC).
%   - gambar    : diisi dengan gambar tandatangan mahasiswa (file tandatangan
%                 bertipe pdf/png/jpg). Dianjukan menggunakan warna biru.
%                 Letakkan gambar di folder "Gambar" dengan nama ttd.jpg/
%                 ttd.png/ttd.pdf (tergantung jenis file. Hapus file ttd.jpg
%                 yang digunakan sebagai contoh
%   - materai   : khusus bagi yang ingin mencetak buku dan menandatangani di 
%                 atas materai. Sama dengan pilihan ``digital'' dan dicetak.
%   - kosong    : tempat kosong ini bisa diisi dengan tanda tangan yang
%                 digambar langsung di atas pdf (fill&sign via acrobat, tanda
%                 tangan dapat dibuat dengan mouse atau stylus)
%=============================================================================
%\ttd{digital}
%\ttd{gambar}
%\ttd{materai}
%\ttd{kosong}
%=============================================================================
\ttd{gambar}

%_____________________________________________________________________________
%=============================================================================
% 								BAGIAN XV
%=============================================================================
% Pilihan tanda tangan digital untuk dosen/pejabat:
%   - no    : pdf TIDAK dapat ditandatangani secara digital, mengakomodasi 
%             yang akan menandatangani via ``menulis'' di file pdf
%   - yes   : pdf dapat ditandatangani secara digital
% 
% PERHATIAN: perubahan ini harus ditanyakan ke kaprodi/dosen koordinator, 
% apakah harus mengisi ``no" atau ``yes". Default = no 
% Untuk mahasiswa Informatika = yes
%=============================================================================
%\ttddosen{yes}
%=============================================================================
\ttddosen{no}

%_____________________________________________________________________________
%=============================================================================
% 								BAGIAN XVI
%=============================================================================
% Tambahkan hyphen (pemenggalan kata) yang anda butuhkan di sini 
%=============================================================================
\hyphenation{ma-te-ma-ti-ka}
\hyphenation{fi-si-ka}
\hyphenation{tek-nik}
\hyphenation{in-for-ma-ti-ka}
%=============================================================================

%_____________________________________________________________________________
%=============================================================================
% 								BAGIAN XVII
%=============================================================================
% Tambahkan perintah yang anda buat sendiri di sini 
%=============================================================================
\renewcommand{\vtemplateauthor}{lionov}
\pgfplotsset{compat=newest}
\setlist{nosep}
%=============================================================================

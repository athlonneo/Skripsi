\documentclass[a4paper,twoside]{article}
\usepackage[T1]{fontenc}
\usepackage[bahasa]{babel}
\usepackage{graphicx}
\usepackage{graphics}
\usepackage{float}
\usepackage[cm]{fullpage}
\pagestyle{myheadings}
\usepackage{etoolbox}
\usepackage{setspace} 
\usepackage{lipsum} 
\setlength{\headsep}{30pt}
\usepackage[inner=2cm,outer=2.5cm,top=2.5cm,bottom=2cm]{geometry} %margin
% \pagestyle{empty}

\makeatletter
\renewcommand{\@maketitle} {\begin{center} {\LARGE \textbf{ \textsc{\@title}} \par} \bigskip {\large \textbf{\textsc{\@author}} }\end{center} }
\renewcommand{\thispagestyle}[1]{}
\markright{\textbf{\textsc{Laporan Perkembangan Pengerjaan Skripsi\textemdash Sem. Genap 2015/2016}}}

\onehalfspacing
 
\begin{document}

\title{\@judultopik}
\author{\nama \textendash \@npm} 

%ISILAH DATA BERIKUT INI:
\newcommand{\nama}{Nicholas Aditya Halim}
\newcommand{\@npm}{2017730018}
\newcommand{\tanggal}{21/06/2021} %Tanggal pembuatan dokumen
\newcommand{\@judultopik}{Implementasi Editor Kode pada SharIF Judge} % Judul/topik anda
\newcommand{\kodetopik}{PAN5001}
\newcommand{\jumpemb}{1} % Jumlah pembimbing, 1 atau 2
\newcommand{\pembA}{Pascal Alfadian}
\newcommand{\pembB}{-}
\newcommand{\semesterPertama}{50 - Genap 20/21} % semester pertama kali topik diambil, angka 1 dimulai dari sem Ganjil 96/97
\newcommand{\lamaSkripsi}{1} % Jumlah semester untuk mengerjakan skripsi s.d. dokumen ini dibuat
\newcommand{\kulPertama}{Skripsi 1} % Kuliah dimana topik ini diambil pertama kali
\newcommand{\tipePR}{B} % tipe progress report :
% A : dokumen pendukung untuk pengambilan ke-2 di Skripsi 1
% B : dokumen untuk reviewer pada presentasi dan review Skripsi 1
% C : dokumen pendukung untuk pengambilan ke-2 di Skripsi 2

% Dokumen hasil template ini harus dicetak bolak-balik !!!!

\maketitle

\pagenumbering{arabic}

\section{Data Skripsi} %TIDAK PERLU MENGUBAH BAGIAN INI !!!
Pembimbing utama/tunggal: {\bf \pembA}\\
Pembimbing pendamping: {\bf \pembB}\\
Kode Topik : {\bf \kodetopik}\\
Topik ini sudah dikerjakan selama : {\bf \lamaSkripsi} semester\\
Pengambilan pertama kali topik ini pada : Semester {\bf \semesterPertama} \\
Pengambilan pertama kali topik ini di kuliah : {\bf \kulPertama} \\
Tipe Laporan : {\bf \tipePR} -
\ifdefstring{\tipePR}{A}{
			Dokumen pendukung untuk {\BF pengambilan ke-2 di Skripsi 1} }
		{
		\ifdefstring{\tipePR}{B} {
				Dokumen untuk reviewer pada presentasi dan {\bf review Skripsi 1}}
			{	Dokumen pendukung untuk {\bf pengambilan ke-2 di Skripsi 2}}
		}
		
\section{Latar Belakang}
SharIF Judge (dengan IF kapital) adalah modifikasi dari aplikasi Sharif Judge buatan Mohammad Javad Naderi yang berfungsi untuk menilai kode program yang diunggah secara otomatis berdasarkan kunci jawaban yang disediakan. SharIF Judge digunakan pada beberapa kuliah di Informatika Unpar untuk mempermudah proses pengumpulan dan penilaian kode program, terutama saat ujian.

Dengan adanya situasi pandemi, seluruh kegiatan kuliah wajib dilaksanakan secara daring. Hal ini menyebabkan berbagai kesulitan, terutama dalam pelaksanaan ujian. Saat ujian sedang berlangsung, umumnya terdapat pengawas yang mengawasi mahasiswa secara fisik untuk mencegah kecurangan. Namun, pengawasan saat ujian menjadi sangat sulit untuk dilakukan saat kuliah dilaksanakan secara daring. Diperlukan sebuah cara untuk merekam tindakan-tindakan mahasiswa selama ujian daring berlangsung. 

Maka, pada skripsi ini akan diimplementasikan editor kode pada SharIF Judge, yang sudah memiliki kemampuan untuk mengompilasi dan menjalankan kode. Dengan demikian, SharIF Judge dapat menjadi sebuah {\it Integrated Development Environment} yang mampu memfasilitasi seluruh proses pembuatan kode serta merekamnya.

{\it Integrated Development Environment} (IDE) adalah sebuah aplikasi yang menyediakan fasilitas untuk pembangunan perangkat lunak. Sebuah IDE memiliki kemampuan untuk mengedit, mengompilasi, dan menjalankan kode program. Pada umumnya, mahasiswa menggunakan aplikasi IDE seperti Netbeans untuk membuat kode program yang kemudian diunggah ke SharIF Judge untuk dinilai.

Dengan mengimplementasikan IDE berbasis web pada SharIF Judge, pengawasan terhadap mahasiswa saat ujian dapat dipermudah dengan merekam ketikan pada editor kode dan mendeteksi bila mahasiswa sedang melihat aplikasi selain SharIF judge.


\section{Rumusan Masalah}
\label{sec:rumusan}
Rumusan masalah yang akan dibahas pada skripsi ini adalah sebagai berikut:
\begin{itemize}
	\item Bagaimana mengimplementasikan {\it Integrated Development Environment} sehingga mahasiswa dapat mengetik dan menjalankan kode dalam SharIF Judge?
\end{itemize}

\section{Tujuan}
\label{sec:tujuan}
Tujuan yang ingin dicapai skripsi ini adalah sebagai berikut:
\begin{itemize}
	\item Mengimplementasikan {\it Integrated Development Environment} sehingga mahasiswa dapat mengetik dan menjalankan kode dalam SharIF Judge.
\end{itemize}


\section{Detail Perkembangan Pengerjaan Skripsi}
Detail bagian pekerjaan skripsi sesuai dengan rencan kerja/laporan perkembangan terkahir :
	\begin{enumerate}
		\item \textbf{Melakukan studi mengenai komponen yang diperlukan untuk membuat IDE berbasis web}\\
		{\bf Status :} Ada sejak rencana kerja skripsi.\\
		{\bf Hasil :} Sebuah IDE berbasis web memerlukan sebuah editor kode dengan fitur program seperti \textit{syntax highlighting}, dan kemampuan untuk mengompilasi dan menjalankan kode yang dibuat.
		
		\item \textbf{Mempelajari struktur SharIF Judge}\\
		{\bf Status :} Ada sejak rencana kerja skripsi.\\
		{\bf Hasil :} SharIF judge dibuat dengan \textit{framework} CodeIgniter. CodeIgniter adalah sebuah \textit{framework} PHP dengan arsitektur MVC (\textit{Model, View, Controller}) untuk mempersingkat penulisan kode dan mempermudah pembangunan web.

		\item \textbf{Merancang IDE berbasis web untuk SharIF Judge}\\
		{\bf Status :} Ada sejak rencana kerja skripsi.\\
		{\bf Hasil :} IDE akan ditambahkan pada halaman Submit dengan kemampuan:
		\begin{itemize}
			\item Melihat soal
			\item Mengetik kode program
			\item Menyimpan dan mengompilasi kode program
			\item Menjalankan kode program dengan input tes kasus sendiri
		\end{itemize}
		
		\item \textbf{Mengimplementasikan IDE pada SharIF Judge}\\
		{\bf Status :} Ada sejak rencana kerja skripsi.\\
		{\bf Hasil :} Sebagian besar fitur sudah berhasil diimplementasikan pada SharIF Judge:
		\begin{itemize}
			\item Melihat soal\\
				Sudah diimplementasikan dengan \textit{library} PDF.js. PDF.js adalah sebuah \textit{library} JavaScript yang dapat menampilkan file PDF dengan memanfaatkan Canvas HTML5.
			\item Mengetik kode program\\
				Sudah diimplementasikan dengan \textit{library} Ace. Ace adalah sebuah \textit{library} JavaScript yang menyediakan editor kode dengan fitur-fitur penulisan kode program seperti \textit{syntax highlighting} dan indentasi.
			\item Menyimpan dan mengompilasi kode program\\
				Fitur ini sudah tersedia sebelumnya pada SharIF Judge, namun hanya untuk kode program yang diupload dalam bentuk file. Modifikasi sudah dilakukan agar kode program dari editor kode dapat disimpan dan dikompilasi.
			\item Menjalankan kode program dengan input tes kasus sendiri\\
				Belum dimplementasikan.
				
		\end{itemize}

		\item \textbf{Melakukan pengujian dan eksperimen}\\
		{\bf Status :} Ada sejak rencana kerja skripsi.\\
		{\bf Hasil :} Akan dilakukan pada Skripsi 2.

		\item \textbf{Menulis dokumen skripsi}\\
		{\bf Status :} Ada sejak rencana kerja skripsi.\\
		{\bf Hasil :} Dokumen skripsi sudah selesai ditulis hingga Bab 1.


	\end{enumerate}

\section{Pencapaian Rencana Kerja}
Langkah-langkah kerja yang berhasil diselesaikan dalam Skripsi 1 ini adalah sebagai berikut:
\begin{enumerate}
	\item Melakukan studi mengenai komponen yang diperlukan untuk membuat IDE berbasis web
	\item Mempelajari struktur SharIF Judge
	\item Merancang IDE berbasis web untuk SharIF Judge
\end{enumerate}



\section{Kendala yang Dihadapi}
%TULISKAN BAGIAN INI JIKA DOKUMEN ANDA TIPE A ATAU C
Kendala - kendala yang dihadapi selama mengerjakan skripsi :
\begin{itemize}
	\item Terlalu banyak melakukan prokratinasi
\end{itemize}

\vspace{1cm}
\centering Bandung, \tanggal\\
\vspace{2cm} \nama \\ 
\vspace{1cm}

Menyetujui, \\
\ifdefstring{\jumpemb}{2}{
\vspace{1.5cm}
\begin{centering} Menyetujui,\\ \end{centering} \vspace{0.75cm}
\begin{minipage}[b]{0.45\linewidth}
% \centering Bandung, \makebox[0.5cm]{\hrulefill}/\makebox[0.5cm]{\hrulefill}/2013 \\
\vspace{2cm} Nama: \pembA \\ Pembimbing Utama
\end{minipage} \hspace{0.5cm}
\begin{minipage}[b]{0.45\linewidth}
% \centering Bandung, \makebox[0.5cm]{\hrulefill}/\makebox[0.5cm]{\hrulefill}/2013\\
\vspace{2cm} Nama: \pembB \\ Pembimbing Pendamping
\end{minipage}
\vspace{0.5cm}
}{
% \centering Bandung, \makebox[0.5cm]{\hrulefill}/\makebox[0.5cm]{\hrulefill}/2013\\
\vspace{2cm} Nama: \pembA \\ Pembimbing Tunggal
}
\end{document}

